\documentclass[../main.tex]{subfiles}

\begin{document}
\section{Introduction}
Turbulence in fluids is an ubiquitous phenomenon in our daily lives, manifesting in everything from the flow of water in a kitchen sink to the swirling smoke of a cigarette. These examples naturally occur in a three-dimensional (3D) space, as it is the intrinsic dimension of our world. However, this thesis focuses on the study of certain aspects of two-dimensional (2D) turbulence, which might initially seem less realistic. Despite this, 2D turbulence holds significant importance in the field of fluid dynamics. For instance, as explained in~\cite{2dturbulence}, many large-scale atmospheric and oceanic phenomena exhibit properties closely related to those observed in 2D turbulence, making such phenomena well-approximated at first order by simplified 2D models.

Theoretically, 2D turbulence behaves quite differently from its 3D counterpart, contrary to natural intuition. While in 3D turbulence, energy cascades from large scales to small scales, in 2D turbulence, energy transfers from small scales to large scales. This fundamental difference is directly related to the primary motivation of this work: understanding the flows in the atmosphere. Although the atmosphere is a 3D domain, it is relatively thin in height, allowing for the observation of phenomena typical of 2D turbulence.

This thesis investigates the behavior of 2D turbulence under the influence of a localized inhomogeneous forcing continuously applied to the system. This forcing is responsible for the continuous generation of vortices in the domain. Several questions arise initially: Will the vortices reach the domain boundaries, or will they dissipate as they spread? How does the energy distribution evolve with distance from the forcing region? These questions are thoroughly addressed in this study. Similar questions were explored in~\cite{alexakis}. In that study
the author considered a long periodic 3D box and found that energy does not reach the domain boundaries if the boundaries are sufficiently large, regardless of the Reynolds number. This work tries to extend that study to the 2D case.

In the present study, a doubly periodic 2D box is analyzed with a localized zero-mean forcing applied to the system. The zero-mean nature of the forcing ensures that no net momentum is injected into the system. Additionally, a point vortex model is examined to compare its results with those obtained from the 2D Navier-Stokes equations.

The structure of this thesis is organized as follows:~\cref{sec:equations} details the theoretical framework of our problem and defines all the relevant quantities.~\cref{sec:results0} describes the numerical setup and presents the results obtained during the study. Finally,~\cref{sec:conclusions} provides conclusions and discusses the implications of our simulations.

\end{document}
