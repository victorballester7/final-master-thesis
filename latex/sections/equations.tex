\documentclass[../main.tex]{subfiles}

\begin{document}
\section{Dynamics of the problem}
As said in the intruduction, the primary problem considered in this work is the integration of the incompressible Navier-Stokes equations with a random forcing term:
\begin{align}
	\partial_t \vf{u} + (\vf{u} \cdot \grad) \vf{u} & = -\grad p + \nu\laplacian \vf{u} + \vf{f} \\
	\divp{\vf{u}}                                   & = 0
\end{align}
where $\vf{u}$ is the velocity field, $p$ is the pressure, $\nu$ is the kinematic viscosity, and $\vf{f}$ is a random forcing term. The forcing term is chosen to be gaussian for the streamfunction formulation, which is detailed below.

\subsection{Streamfunction formulation}
In~\cite{Batchelor2000}, a new variable is introduced in order to simplify the integration of the 2D incompressible Navier-Stokes equations. This quantity, called \emph{streamfunction} and denoted by $\psi$, is defined as:

The forcing term is assumed to be Gaussian, in particular of the form:

\end{document}
